% compila due volte
\documentclass[italian, a4paper]{article}
\newcommand{\TipoDoc}{libro}
%%%%%%%%%%%%%%%% FLAG (per i nuovi utenti: aggiornare le prime 5 flag: \liLaTeXPath \LogoScuolaPath \NomeScuola \CittaScuola \AnnoScolastico) ed eventualmente la flag \PuntiBaseVerifica (dipende da quanti punti base date per la verifica (nel mio caso sono 10/100, quindi è inizializzato a 10)

\providecommand{\liLaTeXPath}{C:/Users/themi/Desktop/LaTeX/liLaTeX} % l'indirizzo dove è stato salvato il file liLaTeX.tex
\providecommand{\LogoScuolaPath}{C:/Users/themi/Desktop/LaTeX/liLaTeX/Logo_ls_Vercelli.jpeg} % il file immagine col logo della scuola
\providecommand{\NomeScuola}{Liceo Scientifico \textit{F.Vercelli}} % il nome della scuola
\providecommand{\CittaScuola}{Asti} % la città in cui si trova la scuola
\providecommand{\AnnoScolastico}{2024-2025} % l'anno scolastico di riferimento (da cambiare una volta all'anno)
\providecommand{\PuntiBaseVerifica}{10} % Numero di punti di base in una verifica. Cancella se non dai nessun punto base
\providecommand{\Data}{} % la data della prova (facoltativa)
\providecommand{\NomeStudente}{} % il nome dello studete con file personale (facoltativo)
\providecommand{\CognomeStudente}{} % il cognome dello studente con file personale (facoltativo)
\providecommand{\NomeClasse}{} % il nome della classe (facoltativo)
\providecommand{\Minuti}{} % il numero di minuti di durata della prova (solo per verifiche)
\providecommand{\Materia}{} % Matematica o Fisica
\providecommand{\Titolo}{} % il titolo del documento
\providecommand{\MarginSize}{1.5} % il margine (in cm). Non settarlo per le immagini
\providecommand{\FontSize}{11} % la dimensione del testo (in pt)
\providecommand{\colText}{black} % il colore di esercizi, teoremi e dimostrazioni
\providecommand{\colSol}{blue} % il colore di dimostrazioni e soluzioni
\providecommand{\colSI}{} % il colore delle unità di misura (definito in seguito come arancione)
\providecommand{\inputTikz}{1} % 0-1 dice se importare anche il file coi comandi tikz
\providecommand{\EnumPages}{1} % può essere posto a 0 se non vogliamo numerare le pagine oppure a 1 se vogliamo numerarle
\providecommand{\TipoDoc}{plain} % il tipo di file. Attualmente supportati: {plain, libro, immagine, verifica, verifica recupero, esercizio svolto, lista esercizi, scheda, scheda SSPM, scheda laboratorio, prova laboratorio, gara olimpiadi, lezione, dimostrazione, formulario}
%NOTA: le flag possono essere sovrascritte dall'utente di volta in volta specificando dei \newcommand in preambolo prima di includere il pacchetto liLaTeX
%Alcune flag, se non specificate dall'utente, vengono ridefinite in questo file nella sezione denominata "SOVRASCRITTURA DELLE FLAG"


%%%%%%%%%%%%%%%% PACCHETTI

\usepackage[T1]{fontenc} % per riconoscere caratteri come è
\usepackage{babel} % supporto per l'italiano (o per le altre lingue)
\usepackage{etoolbox} % if-else commands come \ifdefempty{cs}{T}{F} e \ifdefstring{cs}{string}{T}{F}
\usepackage{amsmath} % comandi matematici
\usepackage{mathrsfs} % per \mathscr
\usepackage{amssymb} % per \varepsilon e simili
\usepackage{amsthm} % definisce \qedsymbol
\usepackage{enumitem} % serve per noitemsep e nolistsep negli itemize e enumerate
\usepackage{multirow} % per l'opzione \multirow{n}{*}{testo} in tabular
\usepackage{mathcomp} % contiene il comando \tccentigrade (per i gradi centrigradi)
\usepackage{eurosym} % contiene il comando \euro (da usare fuori dall'ambiente matematico)
\usepackage{dsfont} % per definire \mathbb{1}, che abbiamo chiamato \bUNO
\usepackage{lmodern} % permette di usare fontsize di misure arbitrarie
\usepackage[fontsize=\FontSize pt]{fontsize} % modifica il font size
\ifdefstring{\TipoDoc}{immagine}{}{\usepackage[margin=\MarginSize cm]{geometry}} % margini
\usepackage{graphicx} % per \includegraphics
\usepackage{caption} % per \caption* e \captionbox
\usepackage[dvipsnames]{xcolor} % pacchetto dei colori
\usepackage{tikz} % tikz è il pacchetto per le immagini. Definisce tikzfigure
\usetikzlibrary{calc, intersections, patterns, decorations.pathmorphing} % per calcoli con tikz, per trovare intersezioni fra curve e per springs e brackets
\usepackage{fancyhdr} % pacchetto per l'impaginazione
\usepackage{adjustbox} % necessario per l'ambiente \adjustbox, utile per disegnare la tabella dei voti
\usepackage{witharrows} % usato per \[\begin{WithArrows}...\end{WithArrows}\] Da includere dopo xcolor
\usepackage{pdfpages} % usato per \includepdf
\usepackage{hyperref} % usato per creare link come \href


%%%%%%%%%%%%%%%% SOVRASCRITTURA DELLE FLAG

% sovrascrive #1 con #2 solo se #1 è vuoto
\newcommand{\OverwriteFlagIfEmpty}[2]{\ifdefempty{#1}{\renewcommand{#1}{#2}}{}}

% in caso di verifica sovrascrive il \Titolo, se vuoto
\ifdefstring{\TipoDoc}{verifica}{\OverwriteFlagIfEmpty{\Titolo}{Verifica di \Materia}}{}
\ifdefstring{\TipoDoc}{verifica recupero}{\OverwriteFlagIfEmpty{\Titolo}{Verifica di recupero di \Materia}}{}
\ifdefstring{\TipoDoc}{prova laboratorio}{\OverwriteFlagIfEmpty{\Titolo}{Prova di laboratorio di \Materia}}{}

% controlla se \TipoDoc è uno dei tipi conosciuti, altrimenti genera un errore di compilazione
\newcounter{TipoDocGuardCounter}
\ifdefstring{\TipoDoc}{plain}{\stepcounter{TipoDocGuardCounter}}{}
\ifdefstring{\TipoDoc}{libro}{\stepcounter{TipoDocGuardCounter}}{}
\ifdefstring{\TipoDoc}{immagine}{\stepcounter{TipoDocGuardCounter}}{}
\ifdefstring{\TipoDoc}{verifica}{\stepcounter{TipoDocGuardCounter}}{}
\ifdefstring{\TipoDoc}{verifica recupero}{\stepcounter{TipoDocGuardCounter}}{}
\ifdefstring{\TipoDoc}{esercizio svolto}{\stepcounter{TipoDocGuardCounter}}{}
\ifdefstring{\TipoDoc}{lista esercizi}{\stepcounter{TipoDocGuardCounter}}{}
\ifdefstring{\TipoDoc}{scheda}{\stepcounter{TipoDocGuardCounter}}{}
\ifdefstring{\TipoDoc}{scheda SSPM}{\stepcounter{TipoDocGuardCounter}}{}
\ifdefstring{\TipoDoc}{scheda laboratorio}{\stepcounter{TipoDocGuardCounter}}{}
\ifdefstring{\TipoDoc}{prova laboratorio}{\stepcounter{TipoDocGuardCounter}}{}
\ifdefstring{\TipoDoc}{gara olimpiadi}{\stepcounter{TipoDocGuardCounter}}{}
\ifdefstring{\TipoDoc}{lezione}{\stepcounter{TipoDocGuardCounter}}{}
\ifdefstring{\TipoDoc}{dimostrazione}{\stepcounter{TipoDocGuardCounter}}{}
\ifdefstring{\TipoDoc}{formulario}{\stepcounter{TipoDocGuardCounter}}{}
\ifnum\value{TipoDocGuardCounter}=0 \errmessage{TipoDoc = \TipoDoc\space non e' un valore valido} \fi % il messaggio di errore di compilazione

% 1 se \TipoDoc è un qualche tipo di verifica, 0 altrimenti
\newcounter{TipoDocVerificaCounter}
\ifdefstring{\TipoDoc}{verifica}{\stepcounter{TipoDocVerificaCounter}}{}
\ifdefstring{\TipoDoc}{verifica recupero}{\stepcounter{TipoDocVerificaCounter}}{}
\ifdefstring{\TipoDoc}{prova laboratorio}{\stepcounter{TipoDocVerificaCounter}}{}

\ifdefstring{\TipoDoc}{immagine}{\renewcommand{\EnumPages}{0}}{} % le immagini non devono mai avere la numerazione delle pagine


%%%%%%%%%%%%%%%% lett

% simile ad enumerate ma con le lettere. Si chiama come \begin{lettfrom}[opzioni]{primo indice}....\end{lettfrom}
\newenvironment{lettfrom}[2][]
{\begin{enumerate}[label=$\mathbf{\alph*.}$,#1]\setcounter{enumi}{#2-1}}
{\end{enumerate}}

% come lettfrom ma parte in automatico dall'indice 1
\newenvironment{lett}[1][]
{\begin{lettfrom}[#1]{1}}
{\end{lettfrom}}


%%%%%%%%%%%%%%%% AMBIENTI

\renewcommand{\qedsymbol}{$\blacksquare$}
\theoremstyle{definition}
\newtheorem{ExEmb}{Esercizio}
\newtheorem*{ExEmb*}{Esercizio}
\newtheorem{ExsecEmb}{Esercizio}[section]
\newtheorem{ProblEmb}[ExEmb]{Problema}
\newtheorem{teoEmb}{Teorema}
\newtheorem*{teoEmb*}{Teorema}
\newtheorem{lemmaEmb}[teoEmb]{Lemma}
\newtheorem*{lemmaEmb*}{Lemma}
\newtheorem{corEmb}[teoEmb]{Corollario}
\newtheorem*{corEmb*}{Corollario}
\newtheorem{SolEmb}{Soluzione}
\newtheorem*{SolEmb*}{Soluzione}
\newtheorem{DimEmb}[SolEmb]{Dimostrazione}
\newtheorem*{DimEmb*}{Dimostrazione}
\theoremstyle{definition}
\newtheorem{Def}{Definizione}
\newtheorem*{Def*}{Definizione}

% questi sono gli ambienti da utilizzare
\newenvironment{Ex}[1][]{\color{\colText}\begin{ExEmb}[#1]}{\end{ExEmb}}
\newenvironment{Ex*}[1][]{\color{\colText}\begin{ExEmb*}[#1]}{\end{ExEmb*}}
\newenvironment{ExSec}[1][]{\color{\colText}\begin{ExSecEmb}[#1]}{\end{ExSecEmb}}
\newenvironment{Probl}[1][]{\color{\colText}\begin{ProblEmb}[#1]}{\end{ProblEmb}}
\newenvironment{teo}[1][]{\color{\colText}\begin{teoEmb}[#1]}{\end{teoEmb}}
\newenvironment{teo*}[1][]{\color{\colText}\begin{teoEmb*}[#1]}{\end{teoEmb*}}
\newenvironment{lemma}[1][]{\color{\colText}\begin{lemmaEmb}[#1]}{\end{lemmaEmb}}
\newenvironment{lemma*}[1][]{\color{\colText}\begin{lemmaEmb*}[#1]}{\end{lemmaEmb*}}
\newenvironment{cor}[1][]{\color{\colText}\begin{corEmb}[#1]}{\end{corEmb}}
\newenvironment{cor*}[1][]{\color{\colText}\begin{corEmb*}[#1]}{\end{corEmb*}}
\newenvironment{Sol}[1][]{\color{\colSol}\begin{SolEmb}[#1]}{\end{SolEmb}}
\newenvironment{Sol*}[1][]{\color{\colSol}\begin{SolEmb*}[#1]}{\end{SolEmb*}}
\newenvironment{Dim}[1][]{\color{\colSol}\begin{DimEmb}[#1]}{\hfill\qedsymbol\end{DimEmb}}
\newenvironment{Dim*}[1][]{\color{\colSol}\begin{DimEmb*}[#1]}{\hfill\qedsymbol\end{DimEmb*}}

\newcommand{\setCounter}[2]{
	\ifthenelse{\equal{#1}{Ex}}{\setcounter{ExEmb}{#2-1}}{}
	\ifthenelse{\equal{#1}{ExSec}}{\setcounter{ExSecEmb}{#2-1}}{}
	\ifthenelse{\equal{#1}{Probl}}{\setcounter{ExEmb}{#2-1}}{}
	\ifthenelse{\equal{#1}{teo}}{\setcounter{teoEmb}{#2-1}}{}
	\ifthenelse{\equal{#1}{lemma}}{\setcounter{teoEmb}{#2-1}}{}
	\ifthenelse{\equal{#1}{cor}}{\setcounter{teoEmb}{#2-1}}{}
	\ifthenelse{\equal{#1}{Sol}}{\setcounter{SolEmb}{#2-1}}{}
	\ifthenelse{\equal{#1}{Dim}}{\setcounter{SolEmb}{#2-1}}{}
}


\newenvironment{ArrowsEq}{\[\begin{WithArrows}}{\end{WithArrows}\]}
% da usare nell'ambiente WithArrows (dentro a \[...\]) (analogo ad align*). Serve compilare due volte.
\newcommand{\Arrowline}[2][cyan]{\Arrow[tikz={font={\mdseries},#1}]{$#2$}\\}


%%%%%%%%%%%%%%%% MATHCAL E SIMILI

\providecommand{\colInc}{red} % colore per le incognite (\u sta per unknown, ovvero incognita in inglese. Infatti il i comandi come \ux verranno utilizzati prevalentemente per evidenziare le incognite nelle equazioni e nei relativi sistemi)
\newcommand{\ua}[1][\colInc]{\textcolor{#1}{\ra}}
\newcommand{\ub}[1][\colInc]{\textcolor{#1}{\rb}}
\newcommand{\uc}[1][\colInc]{\textcolor{#1}{\rc}}
\newcommand{\ud}[1][\colInc]{\textcolor{#1}{\rd}}
\newcommand{\ue}[1][\colInc]{\textcolor{#1}{\re}}
\newcommand{\uf}[1][\colInc]{\textcolor{#1}{\rf}}
\newcommand{\ug}[1][\colInc]{\textcolor{#1}{\rg}}
\newcommand{\uh}[1][\colInc]{\textcolor{#1}{\rh}}
\newcommand{\ui}[1][\colInc]{\textcolor{#1}{\ri}}
\newcommand{\uj}[1][\colInc]{\textcolor{#1}{\rj}}
\newcommand{\uk}[1][\colInc]{\textcolor{#1}{\rk}}
\newcommand{\ul}[1][\colInc]{\textcolor{#1}{\rl}}
\newcommand{\um}[1][\colInc]{\textcolor{#1}{\rm}}
\newcommand{\un}[1][\colInc]{\textcolor{#1}{\rn}}
\newcommand{\uo}[1][\colInc]{\textcolor{#1}{\ro}}
\newcommand{\up}[1][\colInc]{\textcolor{#1}{\rp}}
\newcommand{\uq}[1][\colInc]{\textcolor{#1}{\rq}}
\newcommand{\ur}[1][\colInc]{\textcolor{#1}{\rr}}
\newcommand{\us}[1][\colInc]{\textcolor{#1}{\rs}}
\newcommand{\ut}[1][\colInc]{\textcolor{#1}{\rt}}
\newcommand{\uu}[1][\colInc]{\textcolor{#1}{\ru}}
\newcommand{\uv}[1][\colInc]{\textcolor{#1}{\rv}}
\newcommand{\uw}[1][\colInc]{\textcolor{#1}{\rw}}
\newcommand{\ux}[1][\colInc]{\textcolor{#1}{\rx}}
\newcommand{\uy}[1][\colInc]{\textcolor{#1}{\ry}}
\newcommand{\uz}[1][\colInc]{\textcolor{#1}{\rz}}
\newcommand{\uA}[1][\colInc]{\textcolor{#1}{\rA}}
\newcommand{\uB}[1][\colInc]{\textcolor{#1}{\rB}}
\newcommand{\uC}[1][\colInc]{\textcolor{#1}{\rC}}
\newcommand{\uD}[1][\colInc]{\textcolor{#1}{\rD}}
\newcommand{\uE}[1][\colInc]{\textcolor{#1}{\rE}}
\newcommand{\uF}[1][\colInc]{\textcolor{#1}{\rF}}
\newcommand{\uG}[1][\colInc]{\textcolor{#1}{\rG}}
\newcommand{\uH}[1][\colInc]{\textcolor{#1}{\rH}}
\newcommand{\uI}[1][\colInc]{\textcolor{#1}{\rI}}
\newcommand{\uJ}[1][\colInc]{\textcolor{#1}{\rJ}}
\newcommand{\uK}[1][\colInc]{\textcolor{#1}{\rK}}
\newcommand{\uL}[1][\colInc]{\textcolor{#1}{\rL}}
\newcommand{\uM}[1][\colInc]{\textcolor{#1}{\rM}}
\newcommand{\uN}[1][\colInc]{\textcolor{#1}{\rN}}
\newcommand{\uO}[1][\colInc]{\textcolor{#1}{\rO}}
\newcommand{\uP}[1][\colInc]{\textcolor{#1}{\rP}}
\newcommand{\uQ}[1][\colInc]{\textcolor{#1}{\rQ}}
\newcommand{\uR}[1][\colInc]{\textcolor{#1}{\rR}}
\newcommand{\uS}[1][\colInc]{\textcolor{#1}{\rS}}
\newcommand{\uT}[1][\colInc]{\textcolor{#1}{\rT}}
\newcommand{\uU}[1][\colInc]{\textcolor{#1}{\rU}}
\newcommand{\uV}[1][\colInc]{\textcolor{#1}{\rV}}
\newcommand{\uW}[1][\colInc]{\textcolor{#1}{\rW}}
\newcommand{\uX}[1][\colInc]{\textcolor{#1}{\rX}}
\newcommand{\uY}[1][\colInc]{\textcolor{#1}{\rY}}
\newcommand{\uZ}[1][\colInc]{\textcolor{#1}{\rZ}}
\newcommand{\ra}{\mathrm{a}}
\newcommand{\rb}{\mathrm{b}}
\newcommand{\rc}{\mathrm{c}}
\newcommand{\rd}{\mathrm{d}}
\newcommand{\re}{\mathrm{e}}
\newcommand{\rf}{\mathrm{f}}
\newcommand{\rg}{\mathrm{g}}
\newcommand{\rh}{\mathrm{h}}
\newcommand{\ri}{\mathrm{i}}
\newcommand{\rj}{\mathrm{j}}
\newcommand{\rk}{\mathrm{k}}
\newcommand{\rl}{\mathrm{l}}
\renewcommand{\rm}{\mathrm{m}}
\newcommand{\rn}{\mathrm{n}}
\newcommand{\ro}{\mathrm{o}}
\newcommand{\rp}{\mathrm{p}}
\renewcommand{\rq}{\mathrm{q}}
\newcommand{\rr}{\mathrm{r}}
\newcommand{\rs}{\mathrm{s}}
\newcommand{\rt}{\mathrm{t}}
\newcommand{\ru}{\mathrm{u}}
\newcommand{\rv}{\mathrm{v}}
\newcommand{\rw}{\mathrm{w}}
\newcommand{\rx}{\mathrm{x}}
\newcommand{\ry}{\mathrm{y}}
\newcommand{\rz}{\mathrm{z}}
\newcommand{\rA}{\mathrm{A}}
\newcommand{\rB}{\mathrm{B}}
\newcommand{\rC}{\mathrm{C}}
\newcommand{\rD}{\mathrm{D}}
\newcommand{\rE}{\mathrm{E}}
\newcommand{\rF}{\mathrm{F}}
\newcommand{\rG}{\mathrm{G}}
\newcommand{\rH}{\mathrm{H}}
\newcommand{\rI}{\mathrm{I}}
\newcommand{\rJ}{\mathrm{J}}
\newcommand{\rK}{\mathrm{K}}
\newcommand{\rL}{\mathrm{L}}
\newcommand{\rM}{\mathrm{M}}
\newcommand{\rN}{\mathrm{N}}
\newcommand{\rO}{\mathrm{O}}
\newcommand{\rP}{\mathrm{P}}
\newcommand{\rQ}{\mathrm{Q}}
\newcommand{\rR}{\mathrm{R}}
\newcommand{\rS}{\mathrm{S}}
\newcommand{\rT}{\mathrm{T}}
\newcommand{\rU}{\mathrm{U}}
\newcommand{\rV}{\mathrm{V}}
\newcommand{\rW}{\mathrm{W}}
\newcommand{\rX}{\mathrm{X}}
\newcommand{\rY}{\mathrm{Y}}
\newcommand{\rZ}{\mathrm{Z}}
\newcommand{\cA}{\mathcal{A}}
\newcommand{\cB}{\mathcal{B}}
\newcommand{\cC}{\mathcal{C}}
\newcommand{\cD}{\mathcal{D}}
\newcommand{\cE}{\mathcal{E}}
\newcommand{\cF}{\mathcal{F}}
\newcommand{\cG}{\mathcal{G}}
\newcommand{\cH}{\mathcal{H}}
\newcommand{\cI}{\mathcal{I}}
\newcommand{\cJ}{\mathcal{J}}
\newcommand{\cK}{\mathcal{K}}
\newcommand{\cL}{\mathcal{L}}
\newcommand{\cM}{\mathcal{M}}
\newcommand{\cN}{\mathcal{N}}
\newcommand{\cO}{\mathcal{O}}
\newcommand{\cP}{\mathcal{P}}
\newcommand{\cQ}{\mathcal{Q}}
\newcommand{\cR}{\mathcal{R}}
\newcommand{\cS}{\mathcal{S}}
\newcommand{\cT}{\mathcal{T}}
\newcommand{\cU}{\mathcal{U}}
\newcommand{\cV}{\mathcal{V}}
\newcommand{\cW}{\mathcal{W}}
\newcommand{\cX}{\mathcal{X}}
\newcommand{\cY}{\mathcal{Y}}
\newcommand{\cZ}{\mathcal{Z}}
\newcommand{\bA}{\mathbb{A}}
\newcommand{\bB}{\mathbb{B}}
\newcommand{\bC}{\mathbb{C}}
\newcommand{\bD}{\mathbb{D}}
\newcommand{\bE}{\mathbb{E}}
\newcommand{\bF}{\mathbb{F}}
\newcommand{\bG}{\mathbb{G}}
\newcommand{\bH}{\mathbb{H}}
\newcommand{\bI}{\mathbb{I}}
\newcommand{\bJ}{\mathbb{J}}
\newcommand{\bK}{\mathbb{K}}
\newcommand{\bL}{\mathbb{L}}
\newcommand{\bM}{\mathbb{M}}
\newcommand{\bN}{\mathbb{N}}
\newcommand{\bO}{\mathbb{O}}
\newcommand{\bP}{\mathbb{P}}
\newcommand{\bQ}{\mathbb{Q}}
\newcommand{\bR}{\mathbb{R}}
\newcommand{\bS}{\mathbb{S}}
\newcommand{\bT}{\mathbb{T}}
\newcommand{\bU}{\mathbb{U}}
\newcommand{\bV}{\mathbb{V}}
\newcommand{\bW}{\mathbb{W}}
\newcommand{\bX}{\mathbb{X}}
\newcommand{\bY}{\mathbb{Y}}
\newcommand{\bZ}{\mathbb{Z}}
\newcommand{\bUNO}{\mathds{1}}
\renewcommand{\i}{\imath}
\renewcommand{\j}{\jmath}


%%%%%%%%%%%%%%%% SI

% pone \colSI = black se il testo è da stampare e come orange altrimenti
\definecolor{coppercolor}{RGB}{184, 115, 51} % colore del rame. Usato per colorare le unità di misura nei testi che non vengono stampati
\ifnum\value{TipoDocVerificaCounter}=1 \OverwriteFlagIfEmpty{\colSI}{black} \else
	\ifdefstring{\TipoDoc}{scheda}{\OverwriteFlagIfEmpty{\colSI}{black}}{}
	\ifdefstring{\TipoDoc}{scheda SSPM}{\OverwriteFlagIfEmpty{\colSI}{black}}{}
	\ifdefstring{\TipoDoc}{scheda laboratorio}{\OverwriteFlagIfEmpty{\colSI}{black}}{}
	\ifdefstring{\TipoDoc}{gara olimpiadi}{\OverwriteFlagIfEmpty{\colSI}{black}}{}
	\OverwriteFlagIfEmpty{\colSI}{coppercolor}
\fi

\newcommand{\SI}[1]{\textcolor{\colSI}{\mathrm{#1}}} % scrivi le unità di misura. Per scrive metri scrivi \SI{m} per scrivere joule puoi scrivere \SI{J} oppure \SI{Nm}
\newcommand{\SIf}[2]{\SI{\frac{#1}{#2}}} % unità di misura in frazione. Per scrivere metri al secondo quadrato scrivi \SIf{m}{s^2} oppure \SI{m/s^2}
\newcommand{\ns}[1]{\cdot10^{#1}} % il comando per la notazione scientifica
\newcommand{\oC}{\textcolor{\colSI}{\tccentigrade}} % gradi Celsius


%%%%%%%%%%%%%%%% ALTRE DEFINIZIONI UTILI

% un link. Chiama come \link[colore]{indirizzo}
\newcommand{\link}[2][blue]{\href{#2}{\textcolor{#1}{#2}}}

% tira una riga orizzontale
\newcommand{\riga}{\rule{\linewidth}{0.35mm}}

% parentesi. Chiama come \pare{testo} o come \pare[L]{testo} dove L indica il tipo di parentesi da usare. Le opzioni sono t,q,g,v,f,c
\newcommand\pare[2][t]{
	\ifthenelse{\equal{#1}{q}}{\left[ #2 \right]}{
	\ifthenelse{\equal{#1}{g}}{\left\{ #2 \right\}}{
	\ifthenelse{\equal{#1}{v}}{\left\lvert #2 \right\rvert}{
	\ifthenelse{\equal{#1}{f}}{\left\lfloor #2 \right\rfloor}{
	\ifthenelse{\equal{#1}{c}}{\left\lceil #2 \right\rceil}{
	\left( #2 \right) }}}}}
}

\newcommand{\ang}[3]{#1\widehat{#2}#3} % l'angolo
\newcommand{\bin}[2]{{#1 \choose #2}} % il binomiale di #1 su #2
\newcommand{\vc}[1]{\overrightarrow{#1}} % il vettore (in senso fisico)
\newcommand{\vcp}[3][-5]{\overrightarrow{#2}_{\hspace*{#1pt}#3}} % il vettore (fisico) con pedice

% vettore in senso matematico, con numero arbitrario di input. Chiama come \vett{x}{y}{z}...
% questo è un semplice esempio su come implementare una funzione con un numero arbitrario di input
\makeatletter
\newcommand{\vett}{\begin{pmatrix}\vettIter}
\newcommand{\vettIter}[1]{#1\@ifnextchar\bgroup{\\\vettIter}{\end{pmatrix}}}
\makeatother

\newcommand{\seq}[1][]{\stackrel{#1}{=}} % uguaglianza con commento. Chiama come \seq[commento] (il commento è in ambiente math)
\newcommand{\scong}[1][]{\stackrel{#1}{\cong}} % congruenza con commento. Chiama come \scong[commento] (il commento è in ambiente math)
\newcommand{\lr}[1][]{\stackrel{#1}{\Longrightarrow}} % implicazioni con commento. Chiama come \lr[commento] (il commento è in ambiente math)
\newcommand{\lrt}[1][]{\lr[\text{#1}]} % analogo a \lr, ma il commento è un testo (anziché un ambiente math)

\newcommand{\qsep}[1]{\quad #1 \quad} % separatore fra dei \quad
\newcommand{\qsr}[2]{\qsep{\stackrel{#2}{#1}}} % separatore fra dei \stackrel
\newcommand{\qqe}{\qqsep{\text{e}}} % separa scrivendo un "e" fra le parti separate
\newcommand{\qlr}[1][]{\qsep{\lrt[#1]}} % separa con un'implicazione
% analoghi ai precenti usando \qquad anziché \quad
\newcommand{\qqsep}[1]{\qquad #1 \qquad}
\newcommand{\qqsr}[2]{\qqsep{\stackrel{#2}{#1}}}
\newcommand{\qe}{\qsep{\text{e}}}
\newcommand{\qqlr}[1][]{\qqsep{\lrt[#1]}}


%%%%%%%%%%%%%%%% STRUMENTI PER GEOMETRIA

% ridefiniamo il modo di fare le rette parallele
\makeatletter
\renewcommand{\parallel}{\mathrel{\mathpalette\new@parallel\relax}}
\newcommand{\new@parallel}[2]{%
  \begingroup
  \sbox\z@{$#1T$}% get the height of an uppercase letter
  \resizebox{!}{\ht\z@}{\raisebox{\depth}{$\m@th#1/\mkern-5mu/$}}%
  \endgroup
}

\newcommand{\eqreq}[1][red]{\textcolor{#1}{\textbf{Attenzione!} Per risolvere il problema devi impostare e risolvere un'opportuna equazione o un sistema di equazioni. Soluzioni trovate ``a tentativi'' e solo verificate non sono considerate accettabili.}}

% implicazioni per la geometria. Usa come \Impl[nome teo]{{Hp1}{Hp2}...}{th}
\newcommand{\Impl}[3][]{\[\left.\begin{array}{l}\ImplIter#2\end{array}\right\}\,\lrt[#1]\,\text{#3}\]}
\makeatletter
\newcommand{\ImplIter}[1]{{\text{#1}}\@ifnextchar\bgroup{\\\ImplIter}{}}
\makeatother

% i tre criteri di congruenza e i due criteri di similitudine. Da usare come \cc[nome crit]{Tr1}{Tr2}{Hp1}{Hp2}{Hp3} e \cs[nome crit]{Tr1}{Tr2}{Hp1}{Hp2}
% ad esempio usa come \cc[3]{ABC}{A'B'C'}{$AB \cong A'B'$}{$BC \cong B'C'$}{$CA \cong C'A'$}
\newcommand{\cc}[6][$1^o$c.c.]{Consideriamo i triangoli $#2$ e $#3$:\Impl[#1]{{#4}{#5}{#6}}{$#2 \cong #3$}}
\newcommand{\cs}[5][$1^o$c.s.]{Consideriamo i triangoli $#2$ e $#3$:\Impl[#1]{{#4}{#5}}{$#2 \sim #3$}}


%%%%%%%%%%%%%%%% IMPAGINAZIONE

% crea l'impaginazione per le verifiche
\fancypagestyle{VerificaPageStyle}{%
	\fancyhf{} % resetta impaginazioni residue
    \renewcommand{\headrulewidth}{0pt} % rimuovi la linea in cima alla pagina
    \renewcommand{\footrulewidth}{0pt} % rimuovi la linea in fondo alla pagina
	\pgfmathsetmacro\FontSizeSkip{1.2*\FontSize} % lo spazio verticale da lasciare
    \chead{
    	\ifodd\thepage\fontsize{\FontSize}{\FontSizeSkip}\selectfont
    	\pgfmathsetmacro\NomeCognomeWidthDiffInTab{5} % la differenza di dimensione fra le due colonne di testo nella tabella. Diminuisci se non ci sta il nome della tua scuola
    	\pgfmathsetmacro\NomeCognomeLogoWidth{2} % la dimensione del logo
    	\pgfmathsetmacro\NomeCognomeTabellaWidth{21-2*\MarginSize-\NomeCognomeLogoWidth} % la dimensione della tabella
	\pgfmathsetmacro\NomeCognomeTabellaLeftWidth{.5*\NomeCognomeTabellaWidth-.75+.5*\NomeCognomeWidthDiffInTab} % la larghezza della parte a sinistra della tabella
	\pgfmathsetmacro\NomeCognomeTabellaRightWidth{.5*\NomeCognomeTabellaWidth-.75-.5*\NomeCognomeWidthDiffInTab} % la larghezza della parte a destra della tabella
    	\begingroup\renewcommand\arraystretch{1.5}\begin{center} % lo spazio fra le righe in tabella
    	\Titolo\ifdefempty{\Minuti}{}{\hspace*{1cm}(\textbf{\Minuti} minuti)}\\\vspace{3mm} % la riga con titolo e numero di minuti per la verifica
		\begin{minipage}{\NomeCognomeTabellaWidth cm}
			\begin{center}
			\boxed{
				\begin{tabular}[t]{p{\NomeCognomeTabellaLeftWidth cm}p{\NomeCognomeTabellaRightWidth cm}}% la tabella in cima alle pagine dispari
					COGNOME: \ifdefempty{\CognomeStudente}{\dotfill}{\CognomeStudente} & CLASSE: \NomeClasse \\
					NOME: \ifdefempty{\NomeStudente}{\dotfill}{\NomeStudente} & \NomeScuola \\
					\CittaScuola,\ifdefempty{\Data}{\dotfill}{\Data} & a.s. \AnnoScolastico \\
				\end{tabular}
			}
			\end{center}
		\end{minipage}\begin{minipage}{\NomeCognomeLogoWidth cm} % questa parte deve rimanere su un'unica riga
			\begin{center}\includegraphics[height=2cm]{\LogoScuolaPath}\end{center}
		\end{minipage}
		\end{center}
		\endgroup
    	\fi
    } % intestazione nell'head (solo pagine dispari)
    \cfoot{\ifdefstring{\EnumPages}{0}{}{\fontsize{\FontSize}{\FontSizeSkip}\selectfont\thepage}} % scrivi il numero della pagina in fondo (centrato)
}

\ifnum\value{TipoDocVerificaCounter}=1 % se il file è un file di verifica
\makeatletter % contiene comandi del tipo @, quidni va racchiuso fra \makeatletter e \makeatother
	\pagestyle{VerificaPageStyle}
	% questa parte definisce dove posizionare il testo rispetto all'impaginazione
	\ifdefstring{\EnumPages}{0}{\textheight=25.2cm}{\textheight=24.2cm} % Lo spazio verticale riservato per il testo
	\pgfmathsetmacro\HeaderPts{10*\FontSize-20}
	\pgfmathsetmacro\HeaderPtsDiff{25}
	\pgfmathsetmacro\ExtraHeaderPts{\HeaderPts-\HeaderPtsDiff}
	\patchcmd\@outputpage{\headsep}{\ifodd\count\z@ \HeaderPts pt\else \HeaderPtsDiff pt\fi}{}{}
	\patchcmd\@outputpage{\global\@colht\textheight}{\global\advance\textheight by \ifodd\count\z@ +\ExtraHeaderPts pt \else -\ExtraHeaderPts pt \fi\global\@colht\textheight}{}{}
\makeatother
\else
	\ifdefstring{\EnumPages}{0}{\pagestyle{empty}}{\pagestyle{plain}}
\fi


%%%%%%%%%%%%%%%% TABELLA VOTI

% iteratori utili per \TabellaVoti
\newcounter{TabellaVotiEsNumIter}
\newcounter{TabellaVotiLetterIter}
\newcounter{TabellaVotiPervBarIter}

% assume valori 0-1 per determinare se al termine della colonna attuale dobbiamo disegnare una riga verticale nella prima riga (ovvero se la colonna si riferisce all'ultimo punto dell'attuale esercizio)
\newcommand\TabellaVotiHeaderJump{1}

% Chiama \TabellaVoti a fine verifica per creare la tabella per mettere i punti.
% Il seguente esempio mostra come creare la tabella per una verifica con 5 esercizi (di cui l'es. 3 è diviso in 2 punti e l'es. 5 è diviso in 3 punti) in cui ciascun esercizio vale 18 punti (equamente distribuiti fra i vari punti dell'esercizio):
% \TabellaVoti{{18}{18}{9,9}{18}{6,6,6}}
\newcommand\TabellaVoti[1]{
	\vspace*{\fill} % spostati a fondo pagina
	\pgfmathsetmacro\TabellaVotiTabellaWidth{18}
	\pgfmathsetmacro\TabellaVotiHeight{1}
	\ifdefempty{\PuntiBaseVerifica}{\pgfmathsetmacro\TabellaVotiPtBaseWidth{0}}{\pgfmathsetmacro\TabellaVotiPtBaseWidth{2}}
	\TabellaVotiInizializzaContenitori[\PuntiBaseVerifica]#1 % leggi i dati in input. È importante che sia #1 e non {#1}
	\ifnum\TabellaVotiDrawLettersRow>0 \def\t{2} \else \def\t{1} \fi % decidi se disegnare anche la riga delle lettere
	{\Large\begin{center}
	\makebox[0cm]{ % elimina i margini
	\begin{adjustbox}{minipage=\TabellaVotiTabellaWidth cm, center} % permette di disegnare in un'area estesa orizzontalmente (larga \TabellaVotiTabellaWidth)
	\begin{tikzpicture}
	\pgfmathsetmacro\ltot{\TabellaVotiTabellaWidth-\TabellaVotiPtBaseWidth} % il punto più a destra raggiunto dalla tabella (quello più a sinistra è -\TabellaVotiPtBaseWidth
	\pgfmathsetmacro\l{divide(\ltot,\theTabellaVotiNumLett+1)} % la larghezza di ciascuna colonna (diversa da quelle di Pt.base)
	% disegna le righe orizzontali
	\foreach \y in {-2,...,\t} {\draw (-\TabellaVotiPtBaseWidth,\y*\TabellaVotiHeight) -- ++(\TabellaVotiTabellaWidth,0);}
	\draw[ultra thick] (-\TabellaVotiPtBaseWidth,0) -- ++(\TabellaVotiTabellaWidth,0);
	% le due righe verticali più a sinistra e le due più a destra
	\ifdefempty{\PuntiBaseVerifica}{}{\draw (-\TabellaVotiPtBaseWidth,-2*\TabellaVotiHeight) -- (-\TabellaVotiPtBaseWidth,\t*\TabellaVotiHeight);}
	\draw (\ltot,-2*\TabellaVotiHeight) -- (\ltot,\t*\TabellaVotiHeight);
	\draw[ultra thick] (0,-2*\TabellaVotiHeight) -- (0,\t*\TabellaVotiHeight);
	\draw[ultra thick] (\ltot-\l,-2*\TabellaVotiHeight) -- (\ltot-\l,\t*\TabellaVotiHeight);
	% scriviamo dentro alla prima e all'ultima colonna
	\ifdefempty{\PuntiBaseVerifica}{}{
		\node (W) at (-.5*\TabellaVotiPtBaseWidth,\t*\TabellaVotiHeight-.5*\TabellaVotiHeight) {Pt.base};
		\node (W) at (-.5*\TabellaVotiPtBaseWidth,-.5*\TabellaVotiHeight) {\PuntiBaseVerifica};
		\node (W) at (-.5*\TabellaVotiPtBaseWidth,-1.5*\TabellaVotiHeight) {\PuntiBaseVerifica};
	}
	\node (W) at (\ltot-.5*\l,\t*\TabellaVotiHeight-.5*\TabellaVotiHeight) {\textbf{Tot.}};
	\node (W) at (\ltot-.5*\l,-.5*\TabellaVotiHeight) {\theTabellaVotiPuntiTot};
	% alcuni counter utili per il successivo loop
	\setcounter{TabellaVotiEsNumIter}{1} % un iteratore per tenere traccia dell'attuale esercizio
	\setcounter{TabellaVotiLetterIter}{0} % un iteratore per tenere traccia dell'attuale lettera
	\setcounter{TabellaVotiPervBarIter}{0} % un iteratore per tenere traccia dell'altezza a cui abbiamo tracciato l'ultima barra verticale nella prima riga (dove è iniziato l'attuale esercizio)
	\renewcommand\TabellaVotiHeaderJump{1}
	% le colonne coi punti dei vari esercizi
	\foreach \x in {1,...,\theTabellaVotiNumLett} {
		\stepcounter{TabellaVotiLetterIter} % incrementa l'iteratore della lettera
		\ifnum\csname TabellaVotiBarraPerLettera:\theTabellaVotiNumTabelle:\x\endcsname=1
			% se la colonna attuale è l'ultima dell'attuale esercizio
			\renewcommand\TabellaVotiHeaderJump{1} % la riga verticale deve raggiungere anche la prima riga
			\node (W) at (.5*\value{TabellaVotiPervBarIter}*\l+.5*\x*\l,\t*\TabellaVotiHeight-.5*\TabellaVotiHeight) {Es.\textbf{\theTabellaVotiEsNumIter}}; % scrivi il numero dell'esercizio nella prima riga
			\ifnum\value{TabellaVotiLetterIter}>1
				% se non è l'unica colonna di questo esercizio, scrivi la lettera dell'esercizio (se esiste la riga delle lettere)
				\ifnum\t>0 \node (W) at (\x*\l-.5*\l,.5*\TabellaVotiHeight) {\textbf{\alph{TabellaVotiLetterIter}.}}; \fi
			\fi
			% aggiorna gli iteratori
			\setcounter{TabellaVotiPervBarIter}{\x} % segnamo che abbiamo appena tracciato una riga verticale nella prima riga (ci servirà per stabilire la posizione in cui scrivere il nodo nella prima riga per il prossimo esercizio)
			\stepcounter{TabellaVotiEsNumIter} % scriviamo che siamo al prossimo esercizio
			\setcounter{TabellaVotiLetterIter}{0} % dal momento che siamo in un nuovo esercizio, dobbiamo resettare le lettere
		\else
			% se la colonna attuale NON è l'ultima dell'attuale esercizio, come prima cosa scriviamo la lettera dell'attuale colonna (se esiste la riga delle lettere)
			\ifnum\t>0 \node (W) at (\x*\l-.5*\l,.5*\TabellaVotiHeight) {\textbf{\alph{TabellaVotiLetterIter}.}}; \fi
			\renewcommand\TabellaVotiHeaderJump{0} % la riga verticale NON deve raggiungere la prima riga
		\fi
		\ifnum\x<\theTabellaVotiNumLett
			% disegna la riga verticale a destra dell'attuale esercizio
			\draw (\x*\l,-2*\TabellaVotiHeight) -- ++(0,2*\TabellaVotiHeight+\t*\TabellaVotiHeight-\TabellaVotiHeight+\TabellaVotiHeaderJump*\TabellaVotiHeight);
		\fi
		\node (W) at (\x*\l-.5*\l,-.5*\TabellaVotiHeight) {\getTabellaVotiPuntiPerLettera[\x]}; % scriviamo quanti punti vale l'esercizio nella colonna
		\draw[line width=2pt] (\ltot,-2*\TabellaVotiHeight) rectangle ++(-\l,\TabellaVotiHeight); % la casella in basso a destra viene ripassata
	}
	\end{tikzpicture}
	\end{adjustbox}}
	\end{center}}
	\begin{center}
	{\Huge \setlength\fboxsep{5mm}\boxed{\hspace*{-3mm}\textbf{Voto:} \hspace*{30mm}}} % il box in cui scrivere il voto
	\ifdefstring{\TipoDoc}{verifica recupero}{{\Large\hspace*{10mm}\vspace*{-\baselineskip}
		\begin{tabular}{l}
		Debito saldato:\\
		$\mkern2mu\framebox[16mm]{\huge \textbf{SI}}\mkern15mu\framebox[16mm]{\huge \textbf{NO}}$
		\end{tabular}\\\vspace*{4mm}
	}}{} % per segnare se lo studente ha superato il debito
	\end{center}
}

% valori utili durante la lettura dei dati dall'input per \TabellaVoti
\newcommand\TabellaVotiDrawLettersRow{0} % può essere 0 (falso) o 1 (vero)
\newcounter{TabellaVotiNumLett} % quante lettere ci sono in tutto (se ad es. ci sono es: 1a,1b,2 TabellaVotiNumLett=3)
\newcounter{TabellaVotiPuntiTot} % quanti sono i punti totali per la verifica (idealmente 100)
\newcounter{TabellaVotiNumTabelle} % indica quale tabella di \TabellaVoti stiamo scrivendo (se è la prima, o se ne abbiamo già scritte precedentemente). Questo ci permette di creare più tabelle nello stesso documento.
\newcommand\TabellaVotiInizializzaContenitoriIterOpenNum{0} % usata nella lettura dei dati. Indica se stiamo già leggendo un numero

% funzioni per la gestione dell'array TabellaVotiPuntiPerLettera (quanti punti venono assegnati in ogni colonna della tabella)
\newcommand\storedataTabellaVotiPuntiPerLettera[1][0]{
	\expandafter\newcommand\csname TabellaVotiPuntiPerLettera:\theTabellaVotiNumTabelle:\theTabellaVotiNumLett\endcsname{#1}
}
\newcommand\getTabellaVotiPuntiPerLettera[1][0]{\csname TabellaVotiPuntiPerLettera:\theTabellaVotiNumTabelle:#1\endcsname}

% funzioni per la gestione dell'array TabellaVotiBarraPerLettera (valori 0-1 per decidere se si tratta dell'ultima colonna relativa ad un esercizio)
\newcommand\storedataTabellaVotiBarraPerLettera[1][0]{
	\expandafter\newcommand\csname TabellaVotiBarraPerLettera:\theTabellaVotiNumTabelle:\theTabellaVotiNumLett\endcsname{#1}
}

\makeatletter
% La funzione più esterna nella lettura dell'input per \TabellaVoti
\newcommand\TabellaVotiInizializzaContenitori[1][0]{
	\stepcounter{TabellaVotiNumTabelle}
	\renewcommand\TabellaVotiDrawLettersRow{0}
	\setcounter{TabellaVotiNumLett}{0}
	\ifdefempty{\PuntiBaseVerifica}{\setcounter{TabellaVotiPuntiTot}{0}}{\setcounter{TabellaVotiPuntiTot}{#1}}
	\@ifnextchar\bgroup{\TabellaVotiInizializzaContenitoriIter}{}
}

% Una funzione di iterazione nella lettura dell'input per \TabellaVoti. Viene chiamata una volta per ciascun esercizio. La suddivisione fra gli esercizi viene fatta a "blocchi" (racchiusi fra {}), ma la lettura dentro a queste {} avviene un carattere alla volta
\newcommand\TabellaVotiInizializzaContenitoriIter[1]{
	\TabellaVotiInizializzaContenitoriIterComma,#1
	\@ifnextchar\bgroup{\TabellaVotiInizializzaContenitoriIter}{}
}

% Usata all'interno della lettura dell'input per \TabellaVoti quando leggiamo un separatore (una virgola) dall'input
\newcommand\TabellaVotiInizializzaContenitoriIterComma[2][0]{
	\ifnum#1=1 \renewcommand\TabellaVotiDrawLettersRow{1} \fi
	\renewcommand\TabellaVotiInizializzaContenitoriIterOpenNum{0}
	\stepcounter{TabellaVotiNumLett}
	\TabellaVotiInizializzaContenitoriIterDigit
}

% Usata all'interno della lettura dell'input per \TabellaVoti quando leggiamo una cifra dall'input
\newcommand\TabellaVotiInizializzaContenitoriIterDigit[1]{
	\ifnum\TabellaVotiInizializzaContenitoriIterOpenNum=0
		\storedataTabellaVotiPuntiPerLettera[#1]
		\renewcommand\TabellaVotiInizializzaContenitoriIterOpenNum{1}
	\else
		\expandafter\g@addto@macro\csname TabellaVotiPuntiPerLettera:\theTabellaVotiNumTabelle:\theTabellaVotiNumLett\endcsname{#1}
	\fi
	\@ifnextcharisdigitelse{
		\TabellaVotiInizializzaContenitoriIterDigit
	}{
		\addtocounter{TabellaVotiPuntiTot}{\csname TabellaVotiPuntiPerLettera:\theTabellaVotiNumTabelle:\theTabellaVotiNumLett\endcsname}
		\@ifnextchar,{
			\storedataTabellaVotiBarraPerLettera[0]
			\TabellaVotiInizializzaContenitoriIterComma[1]
		}{
			\storedataTabellaVotiBarraPerLettera[1]
		}
	}
}

% se il prossimo carattere nel buffer è una cifra esegue #1. Altrimenti esegue #2.
\def\@ifnextcharisdigitelse#1#2{%
  \@ifnextchar0{#1}%
  {\@ifnextchar1{#1}%
  {\@ifnextchar2{#1}%
  {\@ifnextchar3{#1}%
  {\@ifnextchar4{#1}%
  {\@ifnextchar5{#1}%
  {\@ifnextchar6{#1}%
  {\@ifnextchar7{#1}%
  {\@ifnextchar8{#1}%
  {\@ifnextchar9{#1}%
  {#2}}}}}}}}}}}
\makeatother



%%%%%%%%%%%%%%%% ERRORI E WARNING

\makeatletter\def\@font@warning#1{}\makeatother % rimuove i warning legati al fontsize


%%%%%%%%%%%%%%%% importa il file tikz
\ifnum\inputTikz>0 % Questo è il pacchetto liLaTeX per tikz

\usepackage{tkz-euclide} % per alcune funzioni grafiche (come \intrette)
\usepackage{pgfplots} % pacchetto per plot

%%%%%%%%%%%%%%%% AMBIENTE immagine

% le immagini vengono generalmente create fra \begin{immagine}[opzioni] e \end{immagine}
\newenvironment{immagine}[1][]
{\begin{figure}[!ht]\centering\begin{tikzpicture}[#1]}
{\end{tikzpicture}\end{figure}}

% analogo ad immagine, ma permette di aggiungere una caption
\newenvironment{immaginecap}[2][]
{\def\immaginecapCaption{#2}\begin{figure}[!ht]\centering\begin{tikzpicture}[#1]}
{\end{tikzpicture}\caption{\immaginecapCaption}\end{figure}}
\newenvironment{immaginecap*}[2][]
{\def\immaginecapCaption{#2}\begin{figure}[!ht]\centering\begin{tikzpicture}[#1]}
{\end{tikzpicture}\caption*{\immaginecapCaption}\end{figure}}


%%%%%%%%%%%%%%%% KEY MANAGER

% da chiamare all'inizio di qualunque comando con dei parametri opzionali come \setkeyfld{parametri opzionali}. Dovrebbe essere sempre la seconda riga di qualunque newcommand con parametri opzionali
\newcommand\setkeyfld[1]{
	% passiamo alla key directory keyfld, e ridefiniamo come #1 tutti le keys specificate in #1
	\tikzset{keyfld/.cd,#1}%
	% definiamo un soprannome \kv per il path alle keys. In questo modo, per riferirci al VALORE di una chiave chiamata K dovremo scrivere \kv{K}, anziché la più lunga /tikz/keyfld/K. Quindi K indica la chiave, mentre \kv{K} indica il suo valore
	\def\kv##1{\pgfkeysvalueof{/tikz/keyfld/##1}}
}

% ifkeyequal{keyname}{value to compare}{do if equal}{do if different}
\newcommand\ifkeyequal[4]{
	\edef\ifkeyequalkey{\kv{#1}} 
	\edef\ifkeyequalcomp{#2}
	\ifdefequal{\ifkeyequalkey}{\ifkeyequalcomp}{#3}{#4}
}
% \ifkeyempty{keyname}{do if empty}{do if non-empty}
\newcommand\ifkeyempty[3]{\ifkeyequal{#1}{\pgfkeysnovalue}{#2}{#3}}

% aggiorna il valore della key #1 al valore #2. Chiama ad esempio come \setkeyvalue{color}{blue} o come \setkeyvalue{h}{\kv{b}}
\newcommand\setkeyvalue[2]{\tikzset{keyfld/.cd,#1=#2}}
% aggiorna il valore della key #1 al valore #2 solo se #1 è empty
\newcommand\setkeyvalueifempty[2]{\ifkeyempty{#1}{\setkeyvalue{#1}{#2}}{}}


%%%%%%%%%%%%%%%% PUNTO

% chiama come \punto[keys]{coordinate}
\newcommand{\punto}[2][]{
	\tikzset{keyfld/.cd,
		col/.initial,
		size/.initial=1.5,
		shape/.initial = circle, % circle, rettangle, diamond o lo si può lasciare vuoto
		name/.initial = puntoNome,
		lbl/.initial,
		lbl ang/.initial = -90,
		lbl dist/.initial = 0.3,
		lbl size/.initial,
		lbl col/.initial,
		col lbl/.initial, % uguale a lbl col (così li accetta entrambi)
	} \setkeyfld{#1};
	\setkeyvalueifempty{col lbl}{\kv{col}};
	\setkeyvalueifempty{lbl col}{\kv{col lbl}};
	\coordinate[\kv{shape},inner sep=\kv{size},fill=\kv{col}] (\kv{name}) at (#2);
	\ifkeyempty{lbl}{}{\node[\kv{lbl col},\kv{lbl size}](puntoLblName) at($(#2)+(\kv{lbl ang}:\kv{lbl dist})$) {\kv{lbl}}};
}

% chiama come \pnt[keys]{nome}{coordinate}
\newcommand{\pnt}[3][]{\punto[name=#2,#1]{#3};}
% chiama come \pntl[keys]{nome}{lbl}{lbl ang}{coordinate}
\newcommand{\pntl}[5][]{\punto[name=#2,lbl=#3,lbl ang=#4,#1]{#5};}
% chiama come \pntle[keys]{name}{lbl ang}{coordinate}
\newcommand{\pntle}[4][]{\pntl[#1]{#2}{$#2$}{#3}{#4};}
% chiama come \pntlp[keys]{name}{lbl ang}{coordinate}
\newcommand{\pntlp}[4][]{\pntl[#1]{#2p}{$#2'$}{#3}{#4};}
% chiama come \pntls[keys]{name}{lbl ang}{coordinate}
\newcommand{\pntls}[4][]{\pntl[#1]{#2s}{$#2''$}{#3}{#4};}


%%%%%%%%%%%%%%%% INTERSEZIONI

% \intpahts{I}{path1}{path2} trova il punto di intersezione fra due paths e lo chiama I
\newcommand{\intpaths}[3]{\path [name intersections={of=#2 and #3,by=#1}];}

% \intrette{I}{A}{B}{C}{D} trova il punto di intersezione I fra le rette AB e CD
\newcommand{\intrette}[5]{\tkzInterLL(#2,#3)(#4,#5) \tkzGetPoint{#1}} \fi
\newcommand{\bs}{\textbackslash}
\newcommand{\ttt}[1]{\texttt{#1}}
\newcommand{\liLaTeX}{\ttt{liLaTeX}}

\title{Manuale del pacchetto \liLaTeX}
\author{Federico Miceli}
\date{estate 2024}

\begin{document}
\maketitle
\tableofcontents

\section{Introduzione}
Questo manuale descrive il pacchetto \liLaTeX. Si tratta di un pacchetto per \LaTeX\ pensato per i docenti di scuola superiore in Italia. Il prefisso \ttt{li} nel nome \liLaTeX infatti rappresenta infatti la parola \textit{liceo}. In questo senso \liLaTeX è il pacchetto LaTeX per i docenti di liceo (o più in generale di scuole superiori italiane).

Il pacchetto è stato sviluppato da Federico Miceli, professore di matematica e fisica del Liceo Scientifico F.Vercelli di Asti, nel corso dell'estate 2024.

\textcolor{Green}{\textbf{Può essere liberamente scaricato (gratuitamente), utilizzato e modificato secondo le proprie esigenze.}}

\textcolor{red}{\textbf{È severamente vietato qualunque utilizzo commerciale e/o a scopo di lucro del pacchetto.}}

\newpage
\section{Come scaricare e inizializzare liLaTeX}\label{installazione}
Il pacchetto \liLaTeX è liberamente scaricabile alla pagina \textit{gitHub} di Federico Miceli.

La pagina in questione può essere trovata all'indirizzo

\begin{center}\link{https://github.com/micelifrc/liLaTeX}\end{center}

È possibile scaricare tutti i file all'interno della Repository, salvando tutti i file in una cartella denominata \liLaTeX all'interno del proprio computer. In alternativa si può decidere di scaricare solo alcuni file fondamentali. All'interno della repository ci sono questi file:
\begin{itemize}
\item Il file \ttt{liLaTeX.tex} è il pacchetto vero è proprio. Devi scaricarlo all'interno della tua cartella \liLaTeX.
\item Il file \ttt{liLaTeXtikz.tex} è il pacchetto per le immagini. Devi scaricarlo e salvarlo all'interno della cartella \liLaTeX. Se non ti interessa la parte grafica del pacchetto puoi evitare di scaricare questo file.
\item I file \ttt{Manuale\_liLaTeX.pdf} e \ttt{Manuale\_liLaTeXtikz.pdf} contengono le istruzioni per i due diversi pacchetti (rispettivamente quello base e quello grafico).
\item La cartella \ttt{Esempi} contiene alcuni semplici file creati con \ttt{liLaTeX.tex}. Non sono file fondamentali, ma può essere utili averli sott'occhio mentre impari a usare il pacchetto \ttt{liLaTeX.tex}
\item La cartella \ttt{Immagini} contiene svariate immagini d'esempio create con \ttt{liLaTeXtikz.tex}. Anche questi file non sono essenziali, ma possono essere utili esempi di utilizzo del pacchetto grafico.
\item Il file \ttt{README} contiene alcune informazioni di base sul pacchetto, ma se stai leggendo questo manuale l'avrai sicuramente già letto.
\end{itemize}

Una volta scaricati i file, è necessario scaricare anche il logo della propria scuola di riferimento, e salvarlo sul computer (generalmente all'interno della cartella \liLaTeX).

Prima di poter utilizzare il pacchetto devi modificare le prime righe di codice del file \ttt{liLaTeX.tex}. Nelle prime righe del file sono infatti definiti alcuni parametri specifici per il modo in cui il file \ttt{liLaTeX.tex} è stato installato sul mio computer. In particolare devi modificare le seguenti righe:
\begin{enumerate}
\item \ttt{\bs providecommand\{\bs liLaTeXPath\}\{C:/Users/themi/Desktop/LaTeX/liLaTeX\}}\\
Sostituire il percorso \ttt{C:/Users/themi/Desktop/LaTeX/liLaTeX} col percorso della propria cartella \liLaTeX.
\item \ttt{\bs providecommand\{\bs LogoScuolaPath\}\{C:/Users/themi/Desktop/LaTeX/liLaTeX/Logo\_ls\_Vercelli.jpeg}\}\\
Sostituire il percorso \ttt{C:/Users/themi/Desktop/LaTeX/liLaTeX/Logo\_ls\_Vercelli.jpeg} con quello del logo della propria scuola.
\item \ttt{\bs providecommand\{\bs NomeScuola\}\{Liceo Scientifico \bs textit\{F.Vercelli\}\}}\\
Sostituire col nome della propria scuola.
\item \ttt{\bs providecommand\{\bs CittaScuola\}\{Asti\}}\\
Sostituire \ttt{Asti} con la città della propria scuola.
\item \ttt{\bs providecommand\{\bs AnnoScolastico\}\{2024-2025\}}\\
Questo idealmente va cambiato all'inizio di ogni anno scolastico.
\item \ttt{\bs providecommand\{\bs PuntiBaseVerifica\}\{10\}}\\
Sostituisci il \ttt{10} col numero di punti di partenza generalmente dati nelle tue verifiche (nel mio caso sono $10/100$). Se non dai punti di partenza cancella il \ttt{10}.
\end{enumerate}

Infine, se hai deciso di non scaricare il file \ttt{liLaTeXtikz.tex}, cerca la riga

\begin{center}\ttt{\bs providecommand\{\bs inputTikz\}\{1\}}\end{center}

all'inizio del file \liLaTeX\ e sostituisci l'\ttt{1} con uno \ttt{0}.

\section{Controllo correttezza inizializzazione}\label{controllo}
Nella cartella \ttt{Esempi} della repository di gitHub è presente un file chiamato

\begin{center}\ttt{verifica\_controllo\_installazione.tex}\end{center}

Prova a scaricarlo e a compilarlo.

Dovresti ottenere un file analogo a quello chiamato \ttt{verifica\_controllo\_installazione.pdf} (sempre nella cartella \ttt{Esempi}).

Se la tua scuola ha un nome molto lungo potresti avere alcuni problema con la tabella a inizio file. Nel caso, cerca la riga

\begin{center}\ttt{\bs pgfmathsetmacro\bs NomeCognomeWidthDiffInTab\{5\}}\end{center}

all'interno del file \liLaTeX\ e prova a sostituire il numero \ttt{5} con altri valori numerici. Quello è il numero che controlla le proporzioni fra le due colonne nella tabella a inizio file nelle verifiche.

\section{Com'è strutturato il file \liLaTeX}
Il file \liLaTeX\ è diviso in diverse sezioni, ciascuna delle quali è più o meno monotematica. Le sezioni sono:
\begin{enumerate}[nolistsep]
\item FLAG
\item PACCHETTI
\item SOVRASCRITTURA DELLE FLAG
\item lett
\item AMBIENTI
\item MATHCAL E SIMILI
\item SI
\item ALTRE DEFINIZIONI UTILI
\item STRUMENTI PER GEOMETRIA
\item IMPAGINAZIONE
\item TABELLA VOTI
\end{enumerate}
Infine, nell'ultima riga del file, viene importato il file \ttt{liLaTeXtikz.tex}.

In questo manuale analizziamo una sezione per volta, descrivendo le funzioni principali di ciascuna di queste sezioni.

\section{FLAG}
La prima sezione del file \liLaTeX\ definisce una serie di FLAG. Oltre a quelle corrette in fase di inizializzazione (già descritte nella sezione \ref{installazione}), sono presenti le seguenti flag, che verranno specificate da un file all'altro:
\begin{itemize}[nolistsep]
\item \ttt{\bs Data}, la data in cui verrà utilizzato il file;
\item \ttt{\bs NomeStudente} o \ttt{\bs CognomeStudente}, specifica il nome e il cognome dello studente a cui è destinato il file (se è destinato a uno studente nello specifico). Se il file è destinato all'intera classe conviene lasciarlo vuoto;
\item i campi \ttt{\bs Materia} e \ttt{\bs Minuti} sono utili per le verifiche. Specificano la materia della verifica e il numero di minuti di durata della stessa;
\item \ttt{\bs Titolo} il titolo del file. Se non specificato, viene inizializzato automaticamente per le verifiche;
\item \ttt{\bs MarginSize} e \ttt{\bs FontSize} sono le dimensioni dei margini (in $\SI{cm}$) e del testo (in \ttt{pt}, che è una misura interna di \LaTeX). \ttt{\bs MarginSize} è normalmente uguale a $1.5$, mentre \ttt{FontSize} è normalmente $11$. Considera comunque che \ttt{\bs MarginSize} viene ignorato (e azzerato) se \ttt{\bs TipoDoc} è \ttt{immagine};
\item \ttt{\bs colText}, \ttt{\bs colSol} e \ttt{\bs colSI} sono rispettivamente il colore con cui scriviamo i testi degli esercizi (o gli enunciati dei problemi), il colore con cui scriviamo le soluzioni dei problemi (o le dimostrazioni dei problemi) e il colore che usiamo per le unità di misura. Normalmente \ttt{\bs colText} è \ttt{black}, \ttt{\bs colSol} è \ttt{blue}, e \ttt{\bs colSI} è un arancione-rame (viene però cambiato in \ttt{black} per i \ttt{\bs TipoDoc} che in genere vengono stampati);
\item \ttt{\bs inputTikz} può avere valore \ttt{0} (falso) oppure \ttt{1} (vero), e dice se vogliamo importare anche il file \ttt{liLaTeXtikz.tex};
\item \ttt{\bs EnumPages} può avere valore \ttt{0} (falso) oppure \ttt{1} (vero). Dice se vogliamo mettere i numeri di pagina nell'impaginazione. Viene resettata come immagine se \ttt{\bs TipoDoc} è uguale a \ttt{immagine} (è l'unico caso in cui una flag viene sovrascritta anche se specificata dall'utente);
\item \ttt{\bs TipoDoc} è il tipo di documento che vogliamo produrre. Sono supportati i tipi di documenti specificati in seguito in questa sezione.
\end{itemize}

Per modificare una flag, è necessario ridefinirla nel file prima in scrivere l'\ttt{\bs input} di \ttt{liLaTeX.tex}.

Ad esempio se vogliamo scrivere che stiamo scrivendo una verifica di fisica, nel preambolo del file (dopo la \ttt{\bs documentclass} ma prima dell'\ttt{\bs input} di \ttt{liLaTeX.tex}) scriveremo le due righe:

\begin{center}
\ttt{\bs newcommand\{TipoDoc\}\{verifica\}}\\
\ttt{\bs newcommand\{Materia\}\{Fisica\}}
\end{center}

\subsection{I valori possibili per la flag \ttt{\bs TipoDoc}}
La flag \ttt{\bs TipoDoc} indica il tipo di documento che stiamo scrivendo. Al momento sono implementati i seguenti tipi di file (anche se non sono del tutto diversi fra loro, al momento):
\begin{itemize}[nolistsep]
\item \ttt{plain} se è un documento senza nulla di particolare (è il valore di default);
\item \ttt{libro} per testi molto lunghi (come questo manuale);
\item \ttt{immagine} se vogliamo semplicemente produrre un'immagine (utile per il pacchetto \ttt{liLaTeXtikz.tex});
\item \ttt{verifica} per le verifiche;
\item \ttt{verifica recupero} per le verifiche di recupero del debito (del primo o del secondo periodo);
\item \ttt{esercizio svolto} per esercizi svolti che vogliamo condividere con gli studenti;
\item \ttt{lista esercizi} per scrivere una lista di esercizi da assegnare (ad esempio per le vacanze estive);
\item \ttt{scheda} per una scheda di lavoro;
\item \ttt{scheda SSPM} per le schede di SSPM (\textit{Scuole Secondarie Potenziate in Matematica}, comunemente chiamato \textit{Liceo Matematico});
\item \ttt{scheda laboratorio} per le schede di laboratorio di fisica;
\item \ttt{prova laboratorio} per le prove (valutate) in laboratorio di fisica;
\item \ttt{gara olimpiadi} per scrivere il testo di una gara (o una simulazione di gara) per le olimpiadi di matematica (me ne occupo personalmente al Vercelli di Asti);
\item \ttt{lezione} per scrivere una lezione o un'UDA (più o meno approfondita). Può poi essere condivisa con gli studenti, o può essere utile a noi come appunti se facciamo una lezione un po' particolare che vogliamo ricordare negli anni successivi o che vogliamo condividere coi colleghi;
\item \ttt{dimostrazione} la dimostrazione di un teorema (o di una lista di teoremi);
\item \ttt{formulario} può essere utilizzata per scrivere piccoli formulari, eventualmente da condividere con gli studenti.
\end{itemize}
La lista potrà ovviamente essere estesa in futuro, man mano che il progetto \ttt{liLaTeX} viene ampliato.

\section{PACCHETTI}
In questa sezione vengono importati tutti i pacchetti esterni. Si tratta di una lista relativamente lunga, ma senza nulla di particolare da dire.

Se l'utente deve aggiungere un pacchetto dovrà di volta in volta valutare se aggiungerlo direttamente al file \ttt{liLaTeX.tex} (se si tratta di un pacchetto che intende utilizzare spesso) oppure se scriverlo solo nel file in cui è utile. Nel secondo caso si suggerisce di includere il pacchetto \textit{dopo} aver importato \ttt{liLaTeX.tex} per ridurre il rischio di conflitti fra pacchetti.

\textbf{Attenzione!} Evita di spostare i pacchetti all'interno del file \ttt{liLaTeX.tex}. Infatti alcuni pacchetti possono entrare fra loro in conflitto se vengono scambiati.

\section{SOVRASCRITTURA DELLE FLAG}
Alcune flag vengono inizializzate in un secondo momento (se non sono state inizializzate dall'utente all'interno del file).

In particolare la flag \ttt{\bs Titolo} viene inizializzata automaticamente nel caso di verifiche (a meno che sia stata fornita dall'utente).

\section{lett}
L'ambiente \ttt{lett} viene usato in modo analogo ad \ttt{enumerate} per le liste puntate. La differenza è che indicizza con le lettere anziché coi numeri.

È possibile specificare le opzioni \textcolor{gray}{\ttt{[noitemsep]}} e \textcolor{gray}{\ttt{[nolistsep]}} (come in \ttt{enumerate}).

Esiste anche l'opzione \ttt{lettfrom} che ci permette di ricominciare l'indicizzazione da un punto successivo.

Per esempio, consideriamo questo codice, che produce questo output (l'output è colorato in blue):\color{gray}\\
\ttt{\bs begin\{lett\}}\\
\ttt{\bs item Prima riga}\\
\ttt{\bs item Seconda riga}\\
\ttt{\bs end\{lett\}}\\
\ttt{Testo intermedio}\\
\ttt{\bs begin\{lettfrom\}\{3\}}\\
\ttt{\bs item Terzo punto}\\
\ttt{\bs end\{lettfrom\}}

\color{blue}
\begin{lett}
\item Prima riga
\item Seconda riga
\end{lett}
Testo intermedio
\begin{lettfrom}{3}
\item Terzo punto
\end{lettfrom}
\color{black}

\section{AMBIENTI}
Gli ambienti possono essere utilizzati per scrivere esercizi, teoremi, dimostrazioni, soluzioni e così via.

Per scrivere un esercizio usiamo la sintassi (il titolo è facoltativo)

\ttt{\bs begin\{Ex\}[titolo]....\bs end\{Ex\}}

Esistono i seguenti ambienti definiti nel pacchetto \ttt{liLaTeX.tex}:
\begin{itemize}[nolistsep]
\item \ttt{Ex} per creare un esercizio;
\item \ttt{Exsec} analogo ad \ttt{Ex}, ma indicizzato sulle sezioni;
\item \ttt{Probl} per creare un problema (simile ad \ttt{Ex});
\item \ttt{Def} per dare una definizione;
\item \ttt{teo} per creare un teorema;
\item \ttt{lemma} per creare un lemma;
\item \ttt{cor} per scrivere un corollario;
\item \ttt{Sol} per scrivere la soluzione di un esercizio;
\item \ttt{Dim} per dare la dimostrazione di un teorema (o risolvere un esercizio dimostrativo).
\end{itemize}

Per la maggior parte di questi ambienti esiste anche una versione equivalente con l'asterisco se non vogliamo numerarli. Ad esempio possiamo usare \ttt{Def*} se non vogliamo dare una numerazione alla definizione.

\subsection{Correzione della numerazione negli ambienti}
A volte potremmo trovare un ambiente numerato in modo \textit{sbagliato} (nel senso che non è quello che vorremmo). In questi casi è possibile cambiare la numerazione dell'ambiente utilizzando il comando \ttt{\bs setCounter} (da usare subito prima di usare l'ambiente).

Ad esempio, se vogliamo resettare la numerazione di \ttt{\bs Sol} in modo che il prossimo ambiente \ttt{\bs Sol} abbia numero \ttt{5} scriveremo il comando \ttt{\bs setCounter\{Sol\}\{5\}}, come qui illustrato (il testo grigio produce l'output blu):\color{gray}\\
\ttt{\bs setCounter\{Sol\}\{5\}}\\
\ttt{\bs begin\{Sol\}}\\
\ttt{\bs Questa è la soluzione del problema.}\\
\ttt{\bs end\{Sol\}}

\setCounter{Sol}{5}
\begin{Sol}
Questa è la soluzione del problema.
\end{Sol}
\color{black}

\subsection{L'ambiente \ttt{ArrowsEq}}
L'ambiente \ttt{ArrowsEq} può essere utilizzato per scrivere un'equazione con delle frecce fra una riga e la successiva, per indicare cosa abbiamo fatto in ciascun passaggio.

La sua sintassi è analoga a quella di \ttt{align*}. Quando andiamo a capo però, anziché usare il comando \ttt{\bs\bs}, possiamo usare il comando \ttt{\bs Arrowline[col]\{...\}} in cui spieghiamo che cosa abbiamo fatto. Considera il seguente esempio (codice in grigio, output in blu):\color{gray}\\
\ttt{\bs begin\{ArrowsEq\}}\\
\hspace*{5mm}\ttt{\&x=x+2\bs Arrowline\{-x\}}\\
\hspace*{5mm}\ttt{\&0=2}\\
\ttt{\bs end\{ArrowsEq\}}\\

\color{blue}
\begin{ArrowsEq}
&x=x+2\Arrowline{-x}
&0=2
\end{ArrowsEq}
\color{black}

\section{MATHCAL E SIMILI}
Per scrivere le lettere in ambienti \ttt{\bs mathcal}, \ttt{\bs mathrm} e \ttt{\bs mathbb} utilizziamo dei comandi abbreviati.

Così un comando come \ttt{\bs rs} scrive la lettera \ttt{s} in \ttt{\bs mathrm}, mentre il comando \ttt{\bs rA} scrive la lettera \ttt{A} in \ttt{\bs mathrm}. In generale, quindi, per scrivere una lettera in \ttt{\bs mathrm} ci basta scrivere la lettera preceduta da \ttt{\bs r}.

In modo analogo, per scrivere una lettera (maiuscola) in \ttt{\bs mathcal} o in \ttt{\bs mathbb} basta scrivere, rispettivamente, \ttt{\bs c} o \ttt{\bs b} prima della lettera. Per il \ttt{\bs mathbb} c'è anche la possibilità di scrivere un $1$, usando il comando \ttt{\bs bUNO}.

Ci sono poi i comandi \ttt{\bs i} e \ttt{\bs j} che producono le lettere $i$ e $j$ senza il puntino.

Infine, possiamo scrivere le lettere in rosso (o nel colore specificato con la flag \ttt{\bs colInc}) per indicare che le lettere così rappresentate sono incognite (unknown). Per farlo, prima della lettera scriviamo \ttt{\bs u}.

Questo è l'elenco dei caratteri che si ottiene in questi modi:
\begin{align*}
&\uA\uB\uC\uD\uE\uF\uG\uH\uI\uJ\uK\uL\uM\uN\uO\uP\uQ\uR\uS\uT\uU\uV\uW\uX\uY\uZ \qquad \ua\ub\uc\ud\ue\uf\ug\uh\ui\uj\uk\ul\um\un\uo\up\uq\ur\us\ut\uu\uv\uw\ux\uy\uz\\
&\rA\rB\rC\rD\rE\rF\rG\rH\rI\rJ\rK\rL\rM\rN\rO\rP\rQ\rR\rS\rT\rU\rV\rW\rX\rY\rZ \qquad \ra\rb\rc\rd\re\rf\rg\rh\ri\rj\rk\rl\rm\rn\ro\rp\rq\rr\rs\rt\ru\rv\rw\rx\ry\rz\i\j\\
&\cA\cB\cC\cD\cE\cF\cG\cH\cI\cJ\cK\cL\cM\cN\cO\cP\cQ\cR\cS\cT\cU\cV\cW\cX\cY\cZ \qquad \bA\bB\bC\bD\bE\bF\bG\bH\bI\bJ\bK\bL\bM\bN\bO\bP\bQ\bR\bS\bT\bU\bV\bW\bX\bY\bZ\bUNO
\end{align*}

\section{SI}
Le unità di misura vengono usate particolarmente spesso in questi file. Per scriverle possiamo usare i comandi \ttt{\bs SI\{...\}} che scriverà le unità di misura nel colore corretto. Se l'unità di misura si presenta come una frazione possiamo invece usare il comando \ttt{\bs SIf\{numeratore\}\{denominatore\}}.

C'è infine il caso dei gradi Celsius, che vengono scritti col comando \ttt{\bs oC}.

Un comando come \ttt{\bs SI\{J/\bs oC kg\}} produce quindi il risultato $\SI{J/\oC kg}$, mentre il comando \ttt{\bs SIf\{m\}\{\bs s\string^2\}} produce il risultato $\SIf{m}{s^2}$.

Il colore utilizzato sarà quello specificato tramite la flag \ttt{\bs colSI} (nero per file da stampare, ramato per quelli da lasciare solo come pdf).

\section{ALTRE DEFINIZIONI UTILI}
In questa sezione sono presenti comandi vari che possono risultare utili per varie ragioni.

Il primo comando interessante è \ttt{\bs link[colore]\{link\}} che genera un link del colore specificato (blu di defalut) che porta all'indirizzo specificato con l'input \ttt{link}.

Il comando \ttt{\bs riga} tira una riga orizzontale nel foglio.

C'è poi il comando \ttt{\bs pare[L]\{INT\}} che crea delle parentesi intorno ad \ttt{INT} del tipo precisato con l'input \ttt{L}. Le parentesi supportate sono:
\begin{itemize}[nolistsep]
\item \ttt{L=t} (default) parentesi tonde;
\item \ttt{L=q} parentesi quadre;
\item \ttt{L=g} parentesi graffe;
\item \ttt{L=v} parentesi verticali (ad esempio per il valore assoluto);
\item \ttt{L=f} floor (arrotondamento per difetto);
\item \ttt{L=c} ceil (arrotondamento per eccesso).
\end{itemize}
Gli output che si ottengono sono quindi
\[\pare[t]{\int e^{-x^2}\,\rd x} \qquad \pare[q]{\int e^{-x^2}\,\rd x} \qquad \pare[g]{\int e^{-x^2}\,\rd x} \qquad \pare[v]{\int e^{-x^2}\,\rd x} \qquad \pare[f]{\int e^{-x^2}\,\rd x} \qquad \pare[c]{\int e^{-x^2}\,\rd x}\]

Il comando \ttt{\bs ang\{A\}\{B\}\{C\}} crea un angolo $\ang{A}{B}{C}$.

Il comando \ttt{\bs bin\{n\}\{k\}} crea un binomiale $\bin{n}{k}$.

I comandi \ttt{\bs vc\{v\}} e \ttt{\bs vcp\{w\}\{1\}} creano i due vettori $\vc{v}$ e $\vcp{w}{1}$.

Il comando \ttt{\bs vett\{a\}\{b\}\{c\}} crea il vettore $\vett{a}{b}{c}$. Questo comando può prendere un numero arbitrario di input (a discrezione dell'utente).

I comandi \ttt{\bs seq}, \ttt{\bs scong} e \ttt{\bs lr} creano dei simboli ($=$, $\cong$ e $\Longrightarrow$) sovrastati dall'argomento opzionale. C'è anche la versione \ttt{\bs lrt} analogo di \ttt{\bs lr}, in cui il commento è scritto fra \ttt{\bs text} (gli altri tre hanno commenti in ambiente matematico).

Quindi i comandi \ttt{\bs lrt[a]}, \ttt{\bs lr[a]}, \ttt{\bs seq[x]} e \ttt{\bs scong[y]} producono gli output $\lrt[a]$, $\lr[a]$, $\seq[x]$ e $\scong[y]$.

Spesso dei comandi analoghi vengono usati come separatori. I più comuni sono \ttt{\bs qe} e \ttt{\bs qlr} (o i loro analoghi \ttt{\bs qqe} e \ttt{\bs qqlr}, in cui viene lasciato maggior spazio). Ad esempio il comando in grigio genera l'output blu:\color{gray}\\
\ttt{x \bs qe y \bs qqlr[a] z \bs qlr t}\\
\color{blue}\[x \qe y \qqlr[a] z \qlr t\]\color{black}

\section{STRUMENTI PER GEOMETRIA}
È comune in problemi di geometri (ma non solo) chiedere esplicitamente (nel testo) di impostare un sistema per risolvere il problema (altrimenti gli studenti potrebbero trovare la soluzione a occhio, e poi lamentare il fatto che non hanno ottenuto i punti del problema, quando tecnicamente hanno trovato la soluzione corretta, e magari l'hanno pure verificata). Per scongiurare questo rischio, in questi problemi conviene usare il comando \ttt{\bs eqreq} che produce il seguente output (in rosso):\\
\eqreq


Spesso nelle dimostrazioni di geometria dobbiamo fare delle implicazioni. Per farlo possiamo usare il comando \ttt{\bs Impl[nome]\{\{Hp1\}\{Hp2\}...\}\{Th\}} (possiamo usare quante ipotesi vogliamo).

Ad esempio il seguente comando (grigio), fuori dall'ambiente matematica, crea il seguente output (blu):\\\color{gray}\\
\ttt{\bs Impl[crit.parall.]\{}\\
\hspace*{5mm}\ttt{\{\$\bs ang\{A\}\{B\}\{C\}\bs cong\bs ang\{D\}\{C\}\{B\}\$\}}\\
\hspace*{5mm}\ttt{\{\$BC\$ trasversale comune\}}\\
\ttt{\}\{\$AB \bs parallel CD\$\}}\\
\color{blue}
\Impl[crit.parall.]{{$\ang{A}{B}{C}\cong\ang{D}{C}{B}$}{$BC$ trasversale comune}}{$AB \parallel CD$}
\color{black}

Dal momento che i criteri di congruenza e di similitudine sono particolarmente comuni, ho scritto anche dei comandi specifici per quelle due famiglie di criteri.

Per i criteri di congruenza la sintassi è \ttt{\bs cc[nome]\{tr1\}\{tr2\}\{Hp1\}\{Hp2\}\{Hp3\}} dove \ttt{nome} è il nome dei criterio utilizzato, \ttt{tr1} e \ttt{tr2} sono i due triangoli fra loro congruenti e \ttt{HP1}, \ttt{HP2}, \ttt{HP3} sono le ipotesi confrontate. Ad esempio, il testo in blu viene generato col comando grigio:\color{gray}\\
\ttt{\bs cc[\$3\string^o\$c.c.]\{ABC\}\{A'B'C'\}\{\$AB \bs cong A'B'\$\}\{\$BC \bs cong B'C'\$\}\{\$CA \bs cong C'A'\$\}}\\
\color{blue}\cc[$3^o$c.c.]{ABC}{A'B'C'}{$AB \cong A'B'$}{$BC \cong B'C'$}{$CA \cong C'A'$}\color{black}

\textbf{Attenzione!} Osserva che le ipotesi vengono scritte in ambiente matematico, ma che i triangoli vengono scritti in testo semplice.

Per i criteri di similitudine la sintassi è identica (si chiama \ttt{\bs cs} anziché \ttt{\bs cc} e usa $2$ ipotesi anziché $3$).

\section{IMPAGINAZIONE}
L'impaginazione cambia a seconda del tipo di documento (specificato con la flag \ttt{\bs TipoDoc}) e dal valore $0-1$ dato alla flag \ttt{\bs EnumPages} (vero di default, a meno che \ttt{\bs TipoDoc=immagine}), oltre ovviamente alla misura dei margini, specificata con la flag \ttt{\bs MarginSize}.

L'impaginazione utilizzata sarà \ttt{empty} oppure \ttt{plain} (a seconda del valore di \ttt{\bs EnumPages}).

Se però il \ttt{TipoDOc} è specifico di una verifica, l'impaginazione è molto più elaborata.

Nel caso, l'impaginazione aggiungerà (in cima alle pagine dispari) un titolo (specificato nella flag \ttt{\bs Titolo}, ma eventualmente arricchito con \ttt{\bs Minuti}).

Poi verrà creata una tabella in cui gli studenti inseriscono i propri dati. Qui compare anche il logo della scuola. Si suggerisce di consultare il file \ttt{\bs verifica\_controllo\_installazione.pdf} (a cui abbiamo già fatto riferimento nella sezione \ref{controllo}), per un esempio sul suo funzionamento.

\section{TABELLA VOTI}
La tabella dei voti deve essere scritta in ogni verifica (generalmente all'inizio della verifica).

La sua sintassi è del tipo \ttt{\bs TabellaVoti\{\{Es1\}\{Es2\}...\}} dove ogni esercizio può a sua volta essere separato in vari punti.

Ad esempio, se abbiamo una verifica in $5$ esercizi, ognuno dei quali vale $18$ punti (e $10$ punti base, specificati con la flag \ttt{\bs PuntiBaseVerifica}) possiamo generare la tabella dei voti col comando\\\textcolor{gray}{\ttt{\bs TabellaVoti\{\{18\}\{18\}\{18\}\{18\}\{18\}\}}}, che genera questa tabella:

\TabellaVoti{{18}{18}{18}{18}{18}}

Se invece il terzo esercizio è diviso in due parti (ciascuna delle quali vale $9$ punti) e il quinto esercizio è diviso in tre parti (ciascuna delle quali vale $6$ punti) la tabella dei voti viene generata col comando \textcolor{gray}{\ttt{\bs TabellaVoti\{\{18\}\{18\}\{9,9\}\{18\}\{6,6,6\}\}}}, che genera questa tabella:

\TabellaVoti{{18}{18}{9,9}{18}{6,6,6}}

\end{document}