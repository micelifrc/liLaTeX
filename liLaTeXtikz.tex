% Questo è il pacchetto liLaTeX per tikz

\usepackage{tkz-euclide} % per alcune funzioni grafiche (come \intrette)
\usepackage{pgfplots} % pacchetto per plot

%%%%%%%%%%%%%%%% AMBIENTE immagine

% le immagini vengono generalmente create fra \begin{immagine}[opzioni] e \end{immagine}
\newenvironment{immagine}[1][]
{\begin{figure}[!ht]\centering\begin{tikzpicture}[#1]}
{\end{tikzpicture}\end{figure}}

% analogo ad immagine, ma permette di aggiungere una caption
\newenvironment{immaginecap}[2][]
{\def\immaginecapCaption{#2}\begin{figure}[!ht]\centering\begin{tikzpicture}[#1]}
{\end{tikzpicture}\caption{\immaginecapCaption}\end{figure}}
\newenvironment{immaginecap*}[2][]
{\def\immaginecapCaption{#2}\begin{figure}[!ht]\centering\begin{tikzpicture}[#1]}
{\end{tikzpicture}\caption*{\immaginecapCaption}\end{figure}}


%%%%%%%%%%%%%%%% KEY MANAGER

% da chiamare all'inizio di qualunque comando con dei parametri opzionali come \setkeyfld{parametri opzionali}. Dovrebbe essere sempre la seconda riga di qualunque newcommand con parametri opzionali
\newcommand\setkeyfld[1]{
	% passiamo alla key directory keyfld, e ridefiniamo come #1 tutti le keys specificate in #1
	\tikzset{keyfld/.cd,#1}%
	% definiamo un soprannome \kv per il path alle keys. In questo modo, per riferirci al VALORE di una chiave chiamata K dovremo scrivere \kv{K}, anziché la più lunga /tikz/keyfld/K. Quindi K indica la chiave, mentre \kv{K} indica il suo valore
	\def\kv##1{\pgfkeysvalueof{/tikz/keyfld/##1}}
}

% ifkeyequal{keyname}{value to compare}{do if equal}{do if different}
\newcommand\ifkeyequal[4]{
	\edef\ifkeyequalkey{\kv{#1}} 
	\edef\ifkeyequalcomp{#2}
	\ifdefequal{\ifkeyequalkey}{\ifkeyequalcomp}{#3}{#4}
}
% \ifkeyempty{keyname}{do if empty}{do if non-empty}
\newcommand\ifkeyempty[3]{\ifkeyequal{#1}{\pgfkeysnovalue}{#2}{#3}}

% aggiorna il valore della key #1 al valore #2. Chiama ad esempio come \setkeyvalue{color}{blue} o come \setkeyvalue{h}{\kv{b}}
\newcommand\setkeyvalue[2]{\tikzset{keyfld/.cd,#1=#2}}
% aggiorna il valore della key #1 al valore #2 solo se #1 è empty
\newcommand\setkeyvalueifempty[2]{\ifkeyempty{#1}{\setkeyvalue{#1}{#2}}{}}


%%%%%%%%%%%%%%%% PUNTO

% chiama come \punto[keys]{coordinate}
\newcommand{\punto}[2][]{
	\tikzset{keyfld/.cd,
		col/.initial,
		size/.initial=1.5,
		shape/.initial = circle, % circle, rettangle, diamond o lo si può lasciare vuoto
		name/.initial = puntoNome,
		lbl/.initial,
		lbl ang/.initial = -90,
		lbl dist/.initial = 0.3,
		lbl size/.initial,
		lbl col/.initial,
		col lbl/.initial, % uguale a lbl col (così li accetta entrambi)
	} \setkeyfld{#1};
	\setkeyvalueifempty{col lbl}{\kv{col}};
	\setkeyvalueifempty{lbl col}{\kv{col lbl}};
	\coordinate[\kv{shape},inner sep=\kv{size},fill=\kv{col}] (\kv{name}) at (#2);
	\ifkeyempty{lbl}{}{\node[\kv{lbl col},\kv{lbl size}](puntoLblName) at($(#2)+(\kv{lbl ang}:\kv{lbl dist})$) {\kv{lbl}}};
}

% chiama come \pnt[keys]{nome}{coordinate}
\newcommand{\pnt}[3][]{\punto[name=#2,#1]{#3};}
% chiama come \pntl[keys]{nome}{lbl}{lbl ang}{coordinate}
\newcommand{\pntl}[5][]{\punto[name=#2,lbl=#3,lbl ang=#4,#1]{#5};}
% chiama come \pntle[keys]{name}{lbl ang}{coordinate}
\newcommand{\pntle}[4][]{\pntl[#1]{#2}{$#2$}{#3}{#4};}
% chiama come \pntlp[keys]{name}{lbl ang}{coordinate}
\newcommand{\pntlp}[4][]{\pntl[#1]{#2p}{$#2'$}{#3}{#4};}
% chiama come \pntls[keys]{name}{lbl ang}{coordinate}
\newcommand{\pntls}[4][]{\pntl[#1]{#2s}{$#2''$}{#3}{#4};}


%%%%%%%%%%%%%%%% INTERSEZIONI

% \intpahts{I}{path1}{path2} trova il punto di intersezione fra due paths e lo chiama I
\newcommand{\intpaths}[3]{\path [name intersections={of=#2 and #3,by=#1}];}

% \intrette{I}{A}{B}{C}{D} trova il punto di intersezione I fra le rette AB e CD
\newcommand{\intrette}[5]{\tkzInterLL(#2,#3)(#4,#5) \tkzGetPoint{#1}}