% flag
\providecommand{\liLaTeXPath}{C:/Users/themi/Desktop/LaTeX/liLaTeX} % l'indirizzo dove è stato salvato il file liLaTeX.tex
\providecommand{\LogoScuolaPath}{C:/Users/themi/Desktop/LaTeX/liLaTeX/Logo_Liceo_Scientifico_Francesco_Vercelli.jpeg} % il file immagine col logo della scuola
\providecommand{\NomeScuola}{Liceo Scientifico \textit{F.Vercelli}} % il nome della scuola
\providecommand{\AnnoScolastico}{2024-2025} % la data della prova (facoltativa)
\providecommand{\Data}{} % la data della prova (facoltativa)
\providecommand{\NomeStudente}{} % il nome dello studete con file personale
\providecommand{\CognomeStudente}{} % il cognome dello studente con file personale
\providecommand{\Minuti}{} % il numero di minuti di durata della prova (solo per verifiche)
\providecommand{\Materia}{} % {Matematica, Fisica}
\providecommand{\Titolo}{} % il nome da dare al file
\providecommand{\TipoDoc}{plain} % il tipo di file. Attualmente supportati: {plain, immagine, verifica, verifica recupero, esercizio svolto, lista esercizi, scheda, scheda laboratorio, prova laboratorio, scheda SSPM}
\providecommand{\MarginSize}{} % il margine (in cm). Sovrascritto in seguito se lasciato vuoto

\usepackage{etoolbox} % if-else commands come \ifdefempty{cs}{TRUE}{FALSE} e \ifdefstring{cs}{string}{TRUE}{FALSE}

% sovrascrive #1 con #2 solo se #1 è vuoto
\newcommand{\overwriteifempty}[2]{\ifdefempty{#1}{\renewcommand{#1}{#2}}{}}

% sovrascrive \MarginSize se è ancora vuoto (a 0 se \TipoDoc = immagine, a 1.5 altrimenti)
\ifdefstring{\TipoDoc}{immagine}{\overwriteifempty{\MarginSize}{0}}{\overwriteifempty{\MarginSize}{1.5}}
% in caso di verifica sovrascrive il titolo, se vuoto
\ifdefstring{\TipoDoc}{verifica}{\overwriteifempty{\Titolo}{Verifica di \Materia}}{}
\ifdefstring{\TipoDoc}{verifica recupero}{\overwriteifempty{\Titolo}{Verifica di recupero di \Materia}}{}
\ifdefstring{\TipoDoc}{prova laboratorio}{\overwriteifempty{\Titolo}{Prova di laboratorio di \Materia}}{}

% pacchetti
\usepackage{graphicx}