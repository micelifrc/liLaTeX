% flag (per i nuovi utenti: aggiornare i primi 4 comandi)
\providecommand{\liLaTeXPath}{C:/Users/themi/Desktop/LaTeX/liLaTeX} % l'indirizzo dove è stato salvato il file liLaTeX.tex
\providecommand{\LogoScuolaPath}{C:/Users/themi/Desktop/LaTeX/liLaTeX/Logo_Liceo_Scientifico_Francesco_Vercelli.jpeg} % il file immagine col logo della scuola
\providecommand{\NomeScuola}{Liceo Scientifico \textit{F.Vercelli}} % il nome della scuola
\providecommand{\AnnoScolastico}{2024-2025} % l'anno scolastico di riferimento (da cambiare una volta all'anno)
\providecommand{\Data}{} % la data della prova (facoltativa)
\providecommand{\NomeStudente}{} % il nome dello studete con file personale (facoltativo)
\providecommand{\CognomeStudente}{} % il cognome dello studente con file personale (facoltativo)
\providecommand{\Minuti}{} % il numero di minuti di durata della prova (solo per verifiche)
\providecommand{\Materia}{} % Matematica o Fisica
\providecommand{\Titolo}{} % il titolo del documento
\providecommand{\MarginSize}{1.5} % il margine (in cm). Non settarlo per le immagini
\providecommand{\FontSize}{11} % la dimensione del testo (in pt)
\providecommand{\TipoDoc}{plain} % il tipo di file. Attualmente supportati: {plain, immagine, verifica, verifica recupero, esercizio svolto, lista esercizi, scheda, scheda SSPM, scheda laboratorio, prova laboratorio, gara olimpiadi, lezione, dimostrazione, formulario}

\usepackage{etoolbox} % if-else commands come \ifdefempty{cs}{TRUE}{FALSE} e \ifdefstring{cs}{string}{TRUE}{FALSE}

% controlla se \TipoDoc è uno dei tipi conosciuti, altrimenti genera un errore di compilazione
\newcounter{TipoDocGuardCounter}
\ifdefstring{\TipoDoc}{plain}{\stepcounter{TipoDocGuardCounter}}{}
\ifdefstring{\TipoDoc}{immagine}{\stepcounter{TipoDocGuardCounter}}{}
\ifdefstring{\TipoDoc}{verifica}{\stepcounter{TipoDocGuardCounter}}{}
\ifdefstring{\TipoDoc}{verifica recupero}{\stepcounter{TipoDocGuardCounter}}{}
\ifdefstring{\TipoDoc}{esercizio svolto}{\stepcounter{TipoDocGuardCounter}}{}
\ifdefstring{\TipoDoc}{lista esercizi}{\stepcounter{TipoDocGuardCounter}}{}
\ifdefstring{\TipoDoc}{scheda}{\stepcounter{TipoDocGuardCounter}}{}
\ifdefstring{\TipoDoc}{scheda SSPM}{\stepcounter{TipoDocGuardCounter}}{}
\ifdefstring{\TipoDoc}{scheda laboratorio}{\stepcounter{TipoDocGuardCounter}}{}
\ifdefstring{\TipoDoc}{prova laboratorio}{\stepcounter{TipoDocGuardCounter}}{}
\ifdefstring{\TipoDoc}{gara olimpiadi}{\stepcounter{TipoDocGuardCounter}}{}
\ifdefstring{\TipoDoc}{lezione}{\stepcounter{TipoDocGuardCounter}}{}
\ifdefstring{\TipoDoc}{dimostrazione}{\stepcounter{TipoDocGuardCounter}}{}
\ifdefstring{\TipoDoc}{formulario}{\stepcounter{TipoDocGuardCounter}}{}
\ifnum\value{TipoDocGuardCounter}=0 \errmessage{TipoDoc = \TipoDoc\space non e' un valore valido} \fi

% sovrascrive #1 con #2 solo se #1 è vuoto
\newcommand{\overwriteifempty}[2]{\ifdefempty{#1}{\renewcommand{#1}{#2}}{}}

% in caso di verifica sovrascrive il titolo, se vuoto
\ifdefstring{\TipoDoc}{verifica}{\overwriteifempty{\Titolo}{Verifica di \Materia}}{}
\ifdefstring{\TipoDoc}{verifica recupero}{\overwriteifempty{\Titolo}{Verifica di recupero di \Materia}}{}
\ifdefstring{\TipoDoc}{prova laboratorio}{\overwriteifempty{\Titolo}{Prova di laboratorio di \Materia}}{}

% pacchetti
\usepackage[T1]{fontenc}
\usepackage{babel}
\usepackage{amsmath} % serve per \text
\usepackage{amsthm} % definisce \qedsymbol
\usepackage{amsfonts} % serve per \mathbb
\usepackage{mathrsfs} % per \mathscr
\usepackage{amssymb} % per \vartriangle e simili
\usepackage[fontsize=\FontSize pt]{fontsize} % modifica il font size
\usepackage{graphicx} % per \includegraphics
\ifdefstring{\TipoDoc}{immagine}{}{\usepackage[margin=\MarginSize cm]{geometry}} % i margini non vanno specificati per le immagini